% DAOconCoq.tex
% Demostracción asistida por ordenador con Coq
% José A. Alonso Jiménez <jalonso@us.es>
% Sevilla, 31 de Julio de 2018 (actualicado el 3 de Agosto)
% =============================================================================

\documentclass[a4paper,12pt,twoside]{book}

%%%%%%%%%%%%%%%%%%%%%%%%%%%%%%%%%%%%%%%%%%%%%%%%%%%%%%%%%%%%%%%%%%%%%%%%%%%%%%
%% § Paquetes adicionales                                                   %%
%%%%%%%%%%%%%%%%%%%%%%%%%%%%%%%%%%%%%%%%%%%%%%%%%%%%%%%%%%%%%%%%%%%%%%%%%%%%%%

% Configuración para XeLaTeX
\usepackage{fontspec}
\usepackage{xltxtra}
\defaultfontfeatures{Ligatures=TeX,Numbers=OldStyle}
\setromanfont{DejaVu Sans}
\setmonofont{DejaVu Sans Mono}[Scale={0.90}]

% Notas: La lista de fuentes disponibles se obtiene con fc-list

\usepackage[spanish]{babel}        % Castellanización.
\usepackage{fancyvrb}              % Verbatim extendido
\usepackage{a4wide}
\usepackage{minted}

\linespread{1.05}                  % Distancia entre líneas
\setlength{\parindent}{2em}        % Indentación de comienzo de párrafo
\raggedbottom                      % No ajusta los espacios verticales.

\usepackage[%
  colorlinks=true,
  urlcolor=blue,
  % pdftex,
  pdfauthor={José A. Alonso <jalonso@us.es>},%
  pdftitle={Demostración asistida por ordenador con Coq},%
  pdfstartview=FitH,%
  bookmarks=false]{hyperref}      

\setcounter{tocdepth}{1}
\setcounter{secnumdepth}{4}

\usepackage{tocstyle}
\usetocstyle{KOMAlike}

%%%%%%%%%%%%%%%%%%%%%%%%%%%%%%%%%%%%%%%%%%%%%%%%%%%%%%%%%%%%%%%%%%%%%%%%%%%%%%
%% § Cabeceras                                                              %%
%%%%%%%%%%%%%%%%%%%%%%%%%%%%%%%%%%%%%%%%%%%%%%%%%%%%%%%%%%%%%%%%%%%%%%%%%%%%%%

\usepackage{fancyhdr}

\addtolength{\headheight}{\baselineskip}

\pagestyle{fancy}

\cfoot{}

\fancyhead{}
\fancyhead[RE]{\mdseries\sffamily DAO con Coq}
\fancyhead[LO]{\mdseries\sffamily \nouppercase{\leftmark}}
\fancyhead[LE,RO]{\mdseries\sffamily \thepage}

%%%%%%%%%%%%%%%%%%%%%%%%%%%%%%%%%%%%%%%%%%%%%%%%%%%%%%%%%%%%%%%%%%%%%%%%%%%%%%
%% § Definiciones                                                           %%
%%%%%%%%%%%%%%%%%%%%%%%%%%%%%%%%%%%%%%%%%%%%%%%%%%%%%%%%%%%%%%%%%%%%%%%%%%%%%%

\input definiciones
\def\mtctitle{Contenido}

%%%%%%%%%%%%%%%%%%%%%%%%%%%%%%%%%%%%%%%%%%%%%%%%%%%%%%%%%%%%%%%%%%%%%%%%%%%%%%
%% § Título                                                                 %%
%%%%%%%%%%%%%%%%%%%%%%%%%%%%%%%%%%%%%%%%%%%%%%%%%%%%%%%%%%%%%%%%%%%%%%%%%%%%%%

\title{{\LARGE Demostración asistida por ordenador con Coq}}
\author{\href{http://www.cs.us.es/~jalonso}
        {\Large José A. Alonso Jiménez}}
\date{\vfill \hrule \vspace*{2mm}
  \begin{tabular}{l}
      \href{http://www.cs.us.es/glc}
           {Grupo de Lógica Computacional} \\
      \href{http://www.cs.us.es}
           {Dpto. de Ciencias de la Computación e Inteligencia Artificial} \\
      \href{http://www.us.es}
           {Universidad de Sevilla}  \\
      Sevilla, 31 de julio de 2018 (versión del 12 de agosto de 2018)
  \end{tabular}\hfill\mbox{}}

%%%%%%%%%%%%%%%%%%%%%%%%%%%%%%%%%%%%%%%%%%%%%%%%%%%%%%%%%%%%%%%%%%%%%%%%%%%%%%
%% § Documento                                                              %%
%%%%%%%%%%%%%%%%%%%%%%%%%%%%%%%%%%%%%%%%%%%%%%%%%%%%%%%%%%%%%%%%%%%%%%%%%%%%%%

% \includeonly{recusion_sobre_numeros_naturales}

% \includexmp{licencia}

\begin{document}

\maketitle
\newpage

\input{licenciaCC}
\newpage

\tableofcontents
\clearpage


\renewcommand{\chaptername}{Tema}

\chapter*{Introducción}

En este libro se incluye unos apuntes de demostración asistida por
ordenador con
\href{https://coq.inria.fr}
     {Coq}
para los cursos de

\begin{itemize}
\item
  \href{http://www.cs.us.es/~jalonso/cursos/m-ra}
       {Razonamiento automático}
  del
  \href{http://master.cs.us.es/Máster_Universitario_en_Lógica,_Computación_e_Inteligencia_Artificial}
       {Máster Universitario en Lógica, computación e inteligencia artificial}
  de la
  \href{http://www.us.es}
       {Universidad de Sevilla}.
\item 
  \href{http://www.cs.us.es/~jalonso/cursos/lmf}
       {Lógica matemática y fundamentos}
  del
  \href{http://www.us.es/estudios/grados/plan_171?p=7}
       {Grado en Matemáticas}
  de la
  \href{http://www.us.es}
       {Universidad de Sevilla}.
\end{itemize}

Esencialmente los apuntes son una adaptación del libro
\href{https://softwarefoundations.cis.upenn.edu/current/lf-current}
     {Software foundations (Vol. 1: Logical foundations)}
de Benjamin Peirce y otros.

Una primera versión de estos apuntes se han usado este año en el
\href{http://www.glc.us.es/~jalonso/SLC2018}
     {Seminario de Lógica Computacional}.

\section*{Cuaderno de bitácora}

En esta sección se registran los cambios realizados en las sucesivas
versiones del libro.

\subsection*{Versión del 12 de agosto de 2018}

Se ha añadido el capítulo 5 (Tácticas básicas de Coq).

\chapter{Programación funcional y métodos elementales de demostración en Coq}
\teoria{T1_PF_en_Coq.v}

\chapter{Demostraciones por inducción sobre los números naturales en Coq}
\teoria{T2_Induccion.v}

\chapter{Datos estructurados en Coq}
\teoria{T3_Listas.v}

\chapter{Polimorfismo y funciones de orden superior en Coq}
\teoria{T4_PolimorfismoyOS.v}

\chapter{Tácticas básicas de Coq}
\teoria{T5_Tacticas.v}

% \appendix
% \include{Examenes}

\end{document}

%%% Local Variables:
%%% mode: latex
%%% TeX-master: t
%%% TeX-engine: xetex
%%% End:
